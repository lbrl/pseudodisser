\section{Введение}


Физика элементарных частиц и их взаимодействий великолепно описывается в рамках Стандартной Модели (СМ),
нашедшей множество экспериментальных подкреплений.
Стоит отметить,
что стандратную модель принято разделять на квантовую электродинамику,
квантовую хромодинамику (КХД)
и слабые взаимодействия.
Однако,
можно выделить два вопроса по отношению к СМ,
важных для данной работы:
возникают проблемы с прямым вычислением в рамках СМ сечений и динамики процессов с участием адронов в области энергий ниже \SI{2}{\GeVr};
с другой стороны,
наблюдается ряд явлений не согласующихся предсказаниями СМ.

Область энергий до \SI{2}{\GeVr} содержит множество адронных состояний,
изучение котороых производится в различных подходах,
среди котороых хотелось бы отметить наиболее универсальный, 
как в смысле числа различных наблюдаемых состояний,
так и достигаемых точностях
---
измерение сечений и динамик реакций на $e^+ e^-$ коллайдерах с универсальным декторами частиц.

Среди противоречивых предсказаний СМ
отдельно стоит выдлеить категорию прицезионных измерений физических величин,
например аномальный магнитный момент мюона $g_\mu - 2$.
Расчёт последнего в рамках СМ требует вычисления адронной поляризации вакуума (АПВ).
В силу вышеупомянутого недостатка КХД,
недостающие данные черпаются из других источников,
кои можно разбить на три категории:
прямое использование экспериментальных данных полученных на $e^+ e^-$ машинах;
разработка феменологических моделей и проведение требуемых вычислений уже в их рамках;
расчёты из первых принципо на решётках.
Стоит отметить,
что можно наблюдать и комбинации этих трёх подходов.
Последняя область является сравнительно молодой и находится в фазе активного роста,
чему способоствует как возросшие доступные вычислительные мощности,
так и развитие подходов.
Первые же два вида вычислений в своей базе имеют те или иные экспериментальные данные,
в основном принадлежщие к одному из традиционно выделяемых классов:
прямые измерения сечений реакций в $e^+ e^-$ столкновениях на сканирующих коллайдерах;
измерения методом радиационного возрата;
данные извлекаемы из распадов $\tau$-мезонов.


Текущая работа посвящена измерению сечения процессов
$e^+e^- \to \pi^0 \gamma$ и 
$e^+e^- \to \eta \gamma$ с последующим распадом псевдоскалярного мезона в $2\gamma$.
Анализ выполнен с детектором КМД-3 на ускорительном комплексе ВЭПП-2000
и исползует \SI{80}{\pbarnr^{-1}},
набранных в диапозоне энергий $\sqrt{s} = \SIrange{0.32}{1.1}{\GeVr}$.



Данные адронных сечений используются для расчёта вклада адронной поляризации вакуума в аномальный магнитный момент мюона $a_\mu$ и константу сильного взаимодействия $\alpha_s(q^2)$.
Предсказания Стандартной Модели для $a_\mu$ расходятся с последним экспериментом по прямому измерения аномального магнитного момента мюона в эксперименте E821 \cite{Bennett:2006fi} на $3.3 \sigma$ в случае расчёта адронной поляризации вакуума по данным $e^+e^-$-аннигиляции \cite{Hagiwara:2011af},
и на $2.4 \sigma$ для определения вклада сильно-взаимодействующих частиц в поляризацию вакуума из результатов изучение распадов $\tau$-мезона \cite{Davier:2010nc}.



Данная работа посвящена измерению сечения процессов $e^+e^- \to \pi^0 \gamma$ и $e^+e^- \to \eta \gamma$ с последующим распадом псевдоскалярного мезона в $2\gamma$.
Данные реакции идут через промежуточные состояния, представленные векторными мезонами.

Радиационные переходы между лёгкими псевдоскалярными и векторными мезонами,
$V \to P\gamma$ и $P\to V \gamma$, 
позволяют изучать смешивание чистых $SU(3)$-состояний кварковой модели, \cite{Feldmann2000}.
Так сравнение теоретических предсказаний и экспериментальных данных позволяет вычислить $\eta - \eta^\prime$ и $\omega - \phi$ углы смешивания, 
для чего определяются эффективные константы взаимодействия $g_{VP\gamma}$.

Изучение радиационных переходов между векторными и псевдоскалярными мезонами интересно для развития теоретической модели Намбу и Иона-Лазинио \cite{Nambu:1961tp}.
Так в работе \cite{Ahmadov2013} рассчитано сечения реакции $e^+e^- \to \eta \gamma$ для энергий до \SI{2}{\GeVr}.
В статье \cite{Arbuzov2011} сечение процесса $e^+e^- \to \pi^0 \gamma$ предсказано для $\sqrt{s}$ вплоть до \SI{2}{\GeVr}.
В данных вычислениях были учтены промежуточные состояния $\rho(770)$, $\omega(782)$, $\phi(1020)$, $\omega(1420)$, $\rho(1420)$ и $\phi(1680)$.
Особый интерес для данной модели представляет изучение сечения $e^+e^- \to \pi^0 \gamma$ вблизи порога реакции.

Радиационный процесс с пионом изучался ранее в работах \cite{CMD2CollaborationPhys.Lett.B605:26-362005}, \cite{Achasov2004}, \cite{Achasov2000a}.
Сечение реакции $e^+e^- \to \eta \gamma$ изучалось в работах \cite{CMD2CollaborationPhys.Lett.B605:26-362005}, \cite{Achasov2013b}, \cite{Achasov2000a}.
В данных работах сечения аппроксимировалось с помощью модели доминантности векторных мезонов, предложенной Сакураи \cite{Sakurai:1960ju}.
Так для данных с энергией в системе центра масс ниже \SI{1.4}{\GeVr} в модели не учитывались возбуждённые состояния $\rho^\prime$, $\omega^\prime$ и $\phi^\prime$.

Задачами данной работы является ра