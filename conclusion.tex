\section{Заключение}

Основная цель работы состоит в измерение сечения реакций
$e^+ e^- \to ( \pi^0 , \, \eta ) \gamma \to 3 \gamma$ по статистике,
набранной детектором КМД-3 на коллайдере ВЭПП-2000.
Для чего был разработан алгоритм выделения событий интересуемых процессов,
включающий в себя процедуру кинематической реконструкции полностью нейтральных процессов,
ориентированную на параметры детектора КМД-3.
Для вычисления сечения в том числе были определены поправка к эффективности нейтрального триггера детектора
и поправка на разность эффективности реконструкции фотонов в эксперименте и моделирование.
Определения первой поправки производится по событиям радиационного Баба рассеяния
$e^+ e^- \to e^+ e^- \gamma$,
второй по событиям $e^+ e^- \to \pi^+ \pi^- \pi^0 \to \pi^+ \pi^- \gamma$.
Получены предварительные результаты,
о которых было доложено на конференциях и опубликованно в их материалах \cite{Ivanov:2018cxt, Akhmetshin:2018mqd}.

В заключение выражаю благодарность Б.\,А. Шварцу, В.\,Е. Шебалину, А.\,Е. Рыжененкову,
а также всему коллективам детектора КМД-3 и ускоритиельных комплексов ВЭПП-2000 и ВЭПП-5 за их слаженную работу,
делающую данные эксперимент и анализ возможными.