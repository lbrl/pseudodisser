\section{Заключение}


В течении всего периода работы детектора КМД -3 систематически проводились калибровки и ремонтное обслуживание электроники катодного тракта Z-камеры.
Разработан алгоритм нахождения и восстановления продольной координаты кластера по катодным полоскам Z-камеры. 
Этот алгоритм добавлен в модуль реконструкции кластера Z-камеры. 
На событиях Баба рассеяния получено пространственное разрешение \~500 мкм.
Разработан алгоритм отделения треков, принадлежащих каонам и пионам, от других треков.
Отлажен алгоритм разделения каонов от пионов в четырёх-трековых и трёх-трековых событиях для процесса .
Определен набор условий для отбора событий этого процесса.
С использованием моделирования по фазовому объему вычислена эффективность регистрации. Построено сечение процесса .
Выделены четыре промежуточных состояния процесса .
Для улучшения точности восстановления импульса трека в программ у добавлен кинематический фит.
В заключение хотелось бы выразить благодарность Г.В. Федотовичу, Е.П. Солодову, А.С. Попову, И.Б. Логашенко, Б.И. Хазин у за полезные советы и помощь.
А тек же, я хочу поблагодарить весь коллектив детектора КМД-3 и комплекса ВЭПП-2000 за их огромный вклад в эксперимент.